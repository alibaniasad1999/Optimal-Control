

\فصل{مقدمه}

چهارپره یا کوادکوپتر\LTRfootnote{Quadcopter} یکی از انواع وسایل پهپاد است. کوادکوپترها نوعی هواگرد بالگردان هستند و در دستهٔ چندپروانه‌ها جای دارند و به دلیل کمک گرفتن از چهار پروانه برای نیروی پیشرانش، به عنوان کواد (چهار) کوپتر نامیده می‌شوند. کوادکوپترها به دلیل داشتن قدرت مانور فوق‌العاده و پروازهایی با تعادل بالا از کاربردهای بسیار گسترده برخوردارند.
در سال‌های اخیر توجه شرکت‌ها، دانشگاه‌ها و مراکز تحقیقاتی بیش از پیش به این نوع از پهپادها جلب شده‌است و لذا روزانه پیشرفت چشم‌گیری در امکانات و پرواز این نوع از پرنده‌ها مشاهده می‌کنیم. کوادکوپترها در زمینه‌های تحقیقاتی، نظامی، تصویر برداری و تفریحی و سمپاشی از کاربرد بالا و روزافزونی برخوردارند و مدل‌های دارای سرنشین آن نیز تولید شده‌است.



\قسمت{ساختار}

کوادکوپترها همانند انواع دیگر وسایل پرنده از ایجاد اختلاف فشار در اتمسفر پیرامون خود برای بلند شدن و حرکت در هوا استفاده می‌نمایند. همان‌طور که هلیکوپتر‌ها به کمک پره اصلی این اختلاف فشار را ایجاد می‌کنند و نیروی برآ ی خود را تأمین می‌کنند. در هلیکوپترها به دلیل وجود نیروی عمل و عکس‌العمل، پس از اینکه پره اصلی شروع به چرخش می‌کند با برخورد مولکول‌های هوا به این پره و وجود عکس‌العمل، یک نیرویی با جهت مخالف جهت چرخش پره به پره و در ادامه به شفت متصل به پره اعمال می‌شود (نیروی گشتاور) و باعث چرخش هلیکوپتر به دور خود می‌شود. حالا برای حل این مشکل از پره دم هلیکوپتر استفاده می‌شود تا نیرویی را تولید کند که مانع چرخش هلیکوپتر به دور خود شود. حال اگر هلیکوپتر به جای داشتن یک پره اصلی از دو پره اصلی که خلاف جهت یکدیگر بچرخند استفاده می‌نمود، به دلیل خنثی شدن دو نیروی گشتاور توسط یکدیگر، دیگر هلیکوپتر به دور خود نمی‌چرخید. مانند هلیکوپترهای شینوک\LTRfootnote{Boeing CH-47 Chinook}. حال با توجه به توضیحات داده شده راحت‌تر می‌توان به ساختار کوادکوپترها اشاره نمود.
\begin{figure}[H]
	\includegraphics[width=12cm]{figs/introduction/boeing-ch-chinook.jpg}
	\centering
	\caption{هلیکوپتر شینوک\cite{CH-47}}
\end{figure}


کوادکوپترها با بهره‌گیری از چهار موتور و پره مجزا و چرخش دو به دو معکوس این موتورها نیروی گشتاورهای ایجاد شده را خنثی می‌کنند و همچنین اختلاف فشار لازم جهت ایجاد نیروی برآ را تأمین می‌کنند.

\begin{figure}[H]
	\includegraphics[width=12cm]{figs/introduction/Quadblade.jpg}
	\centering
	\caption{چرخش پره‌های چهارپره\cite{Quadhowfly}}
\end{figure}
نحوه ایجاد فرامین کنترلی در کوادکوپترها به این صورت است که، برای تغییر ارتفاع از کم یا زیاد کردن سرعت چرخش همه موتورها استفاده می‌شود و باعث کمتر یا زیاد تر شدن نیروی برآ می‌شود. برای چرخش کوادکوپتر به دور خود و به صورت درجا، دو پره هم جهت با سرعت کمتر و دو پره هم جهت دیگر با سرعت بیشتر می‌چرخند و نیروی گشتاور به یک سمت ایجاد می‌شود و نیرویه برآ همانند قبل است (زیرا دو پره با سرعت کمتر و دو پره دیگر به همان نسبت با سرعت بیشتر می‌چرخند) لذا کوادکوپتر در ارتفاع ثابت به دور خود می‌چرخد. برای حرکت کوادکوپترها در جهت‌های مختلف (عقب، جلو، چپ و راست) توسط کم و زیاد کردن سرعت موتورها کوادکوپتر را از حالت افقی خارج کرده و باعث حرکت آن می‌شوند.

\قسمت{تاریخچه}
مدل‬ اولیه آزمایشی یک چندموتوره\LTRfootnote{Multiroter} در سال ۱۹۰۷ توسط دو برادر فرانسوی بنام \lr{Jacques and Louis Breguet} در پروژه‌ای بنام Quadcopter ساخته و تست شد، هرچند آن‌ها نتوانستند پرنده خود را در آسمان نگه دارند ولی موفق به پرواز ثابت شدند. بعد از آن ساخت بالگرد چهار پروانه‌ای به سال ۱۹۲۰ میلادی برمیگردد. در آن سال یک مهندس فرانسوی بنام \lr{etienne oehmichen} اولین بالگرد چهارپره را اختراع نمود و مسافت ۳۶۰ متر را با کوادکوپتر خود پرواز کرد در همان سال او مسافت یک کیلومتر را در مدت هفت دقیقه و چهل ثانیه پرواز کرد.

در حدود سال ۱۹۲۲ در آمریکا \lr{Dr George de Btheza} موفق به ساخت و تست تعدادی Quadcopter برای ارتش شد که قابلیت کنترل و حرکت در سه بعد را داشت، ولی پرواز با آن بسیار سخت بود.

در سال‌های اخیر توجه مراکز دانشگاهی به طراحی و ساخت پهپادهای چهارپره جلب شده‌است و مدل‌های مختلفی در دانشگاه استنفورد و کورنل ساخته شده‌است و به تدریج رواج یافته‌است~\cite{5717652}.

از حدود سال ۲۰۰۶ کواد کوپترها شروع به رشد صنعتی به صورت وسایل پرنده بدون سرنشین نمودند.


\قسمت{تعریف مسئله}
مسئلهای که در این پروژه بررسی می‌شود، کنترل وضعیت سه درجه آزادی استند آزمایشگاهی چهارپره با استفاده از روش کنترل خطی مبتنی بر روش بازی دیفرانسیلی است. این استند آزمایشگاهی شامل یک چهارپره است که از 
مرکر توسط یک اتصال به یک پایه وصل شده‌است. در این صورت، تنها وضعیت (زوایای رول، پیچ و یاو) 
چهارپره تغییر کرده و فاقد حرکت انتقالی است. همچنین میتوان با مقیدکردن چرخش حول هر محور ، 
حرکات رول، پیچ و یاو پرنده را به صورت مجرا و با یکدیگر بررسای کارد.
استند آزمایشگاهی سه درجه آزادی چهارپره در شکل \ref{LabQuad} نشان داده شده‌است.

\begin{figure}[H]\label{LabQuad}
	\includegraphics[width=12cm]{figs/introduction/3DOFQuad.jpg}
	\centering
	\caption{استند سه درجه آزادی چهارپره آزمایشگاه\cite{Iranlabexpo}}
\end{figure}
با توجه به شکل مرکز جرم این استند بالاتر از مفصل قرار دارد که می‌توان به صورت آونگ معکوس در نظر گرفت. بنابراین سیستم بدون جضور کنترل کننده ناپایدار است. این سیستم دارای چهار ورودی مستقل (سرعت چرخش پره‌ها) و سه خروجی زاوای اویلر ($\psi, \theta, \phi$) است. در مدل سازی این استند عدم قطعیت وجود دارد، اما با توجه به کنترل کننده مورد استفاده می‌توان این عدم قطعیت را به صورت اغتشاش در نظر گرفت و سیستم را به خوبی کنترل کرد. در پایان این کنترل کننده را با کنترل کننده تناسابی - انتگرالای -
مشتقی (PID) مقایسه خواهد‌شد.




\قسمت{پیشینه پژوهش}
این بخش را می‌توان به دو قسمت پیشینه کارهای انجام شاده در حوزه کنترل چهارپره و استفاده از تئوری بازی
در کنترل تقسیم‌بندی کرد.
\subsection{کنترل‌کننده مبتنی بر تئوری بازی}
تئوری بازی با استفاده از مدل‌های ریاضی به تحلیل روش‌های همکاری یا رقابت موجودات منطقی و هوشمند می‌پردازد. تئوری بازی، شاخه‌ای از ریاضیات کاربردی است که در علوم اجتماعی و به ویژه در اقتصاد، زیست‌شناسی، مهندسی، علوم سیاسی، روابط بین‌الملل، علوم رایانه، بازاریابی، فلسفه و قمار مورد استفاده قرار می‌گیرد. نظریهٔ بازی در تلاش است تا بوسیلهٔ ریاضیات، رفتار را در شرایطِ راهبردی یا در یک بازی که در آن‌ها موفقیتِ فرد در انتخاب کردن، وابسته به انتخاب دیگران می‌باشد، برآورد کند.
\subsubsection{تاریخچه تئوری بازی}
در سال ۱۹۲۱ یک ریاضی‌دان فرانسوی به نام اِمیل بُرِل برای نخستین بار به مطالعهٔ تعدادی از بازی‌های رایج در قمارخانه‌ها پرداخت و چند مقاله در موردِ آن‌ها نوشت. او در این مقاله‌ها بر قابل پیش‌بینی بودنِ نتایجِ این نوع بازی‌ها از راه‌های منطقی، تأکید کرده بود. در سال ۱۹۹۴ جان فوربز نش به همراهِ جان هارسانی و راینهارد سیلتن به خاطر مطالعات خلاقانه خود در زمینهٔ نظریهٔ بازی، برندهٔ جایزه نوبل اقتصاد شدند. در سال‌های پس از آن نیز بسیاری از برندگانِ جایزه‌ی نوبل اقتصاد از میانِ متخصصینِ تئوری بازی انتخاب شدند. آخرینِ آنها، ژان تیرول فرانسوی است که در سال ۲۰۱۴ این جایزه را کسب کرد.
\subsubsection{تعادل نش}
در تئوری بازی، تعادل نش (به نام جان فوربز نش، که آن را پیشنهاد کرد) راه حلی از تئوری بازی است که شامل دو یا چند بازیکن، که در آن فرض بر آگاهی هر بازیکن به استراتژی تعادل بازیکنان دیگر است و بدون هیچ بازیکنی که فقط برای کسب سود خودش با تغییر استراتژی یک جانبه عمل کند. اگر هر بازیکنی استراتژی را انتخاب کند هیچ بازیکنی نمی‌تواند با تغییر استراتژی خود در حالی که نفع بازیکن دیگر را بدون تغییر نگه داشته باشد عمل کند، سپس مجموعه انتخاب‌های استراتژی فعلی و بهره‌مندی مربوطه، تعادل نش را تشکیل می‌دهد.
%\begin{figure}[H]\label{LabQuad}
%\includegraphics[width=6cm]{figs/introduction/John_Forbes_Nash,_Jr._by_Peter_Badge.jpg}
%\centering
%\caption{جان فوربز نش\cite{JanNash}}
%\end{figure}
\subsubsection{تئوری بازی در کنترل سیستم}
در منبع \cite{article1} به صورت خلاصه نظریه بازی و تعادل نش توضیح داده شده‌است. در حالتی از تئوری بازی می‌توان با دیگر بازیکنان همکاری یا رقابت کردکه در منیع\cite{8376282} یرای یک پهباد\LTRfootnote{UAC(unmanned aerial vehicle)} بررسی شده است. به علت اینکه هدف هر بازیکن افزایش امتیاز خود و کاهش امتیاز رقیب هست این مسئله از نوع غیرمشارکتی\LTRfootnote{non-cooperative} در نظر گرفته شده‌است. در این مسئله دو معادله دیفرانسیل شروع به بازی با یکدیگر می‌کنند که هدف هر کدام افزایش امتیاز خود است. در روش کنترل کننده خطی مبتنی بر روش بازی دیفرانسیلی یک تابع هزینه برای هر بازیکن ایجاد می‌شود. این مسائل فرض شده که اطلاعات در اختیار تمامی بازیکنان قرار دارد و هیچ یک از آینده خبر ندارند.

از کابردهای بازی دیفرانسیلی می‌توان به فرود بر روی اجسام متحرک مانند فرود چهارپره، هلیکوپتر و پهباد بر روی ناو\cite{8996044} اشاره کرد. در منبع \cite{9001045} از تئوری بازی و بازی دیفرانسیلی برای نبرد بین دو پهباد استفاده شده‌است. قدرت تئوری بازی بر تحلیل رفتارهای دو یا چندین بازیکن است بر همین اساس در منبع \cite{Pachter2019} برای دفاع و بررسی تحدید بازیکنان دیگر و در منبع \cite{7502594} از کنترل کننده خطی برای شکل پرواز گروه سه نفره از پهبادها استفاده شده‌است. بازی دیفرانسیلی در ناوبری کاردبرد ویژه‌ای دارد، در منبع \cite{6160198} از این روش برای هدایت و ناوبری یک میکروپهباد\LTRfootnote{Micro-UAV (MAV)} استفاده شده‌است. در منبع \cite{1595165} از بازی دیفرانسیلی برای گشت و گریز پهباها استفاده شده‌است.

تابع هزینه در این مسائل بسیار شبیه به کنترل‌کننده بهینه خطی\LTRfootnote{LQR} است. در منبع \cite{4399042} از روش کنترل‌کننده بهینه خطی برای کنترل وضعیت یک چهارپره استفاده شده‌است. در منبع \cite{article2} شراط وجود جواب و حل معادلات ریکاتی \LTRfootnote{Riccati} LQDG ارائه شده‌است. در این کنترل‌کننده احتیاج به مدل سیستم است.
\subsection{کنترل چهارپره}
