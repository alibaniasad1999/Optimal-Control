
% -------------------------------------------------------
%  Abstract
% -------------------------------------------------------


\pagestyle{empty}

\begin{وسط‌چین}
\مهم{چکیده}
\end{وسط‌چین}

\بدون‌تورفتگی در این پژوهش از یک روش مبتنی بر تئوری بازی\LTRfootnote{Game Theory} استتفاده شده‌است. در این روش سیستم و اغتشاش دو بازیکن اصلی در نظر گرفته می‌شود. هر یک سعی می‌کنند امتیاز خود را  با کمترین هزینه افزایش دهند که در اینجا امتیاز، وضعیت استند در نظر گرفته ‌شده‌است. در این روش انتخاب حرکت با استفاده از معادله نش\LTRfootnote{Nash Equilibrium}
 که هدف آن کم کردن تابع هزینه با فرض بدترین حرکت دیگر بازیکن، انجام می‌شود.
این روش نسبت به اغتشاش خارجی و 
نویز سنسور مقاوم است. همچنین نسبت به عدم قطعیت مدلسازی نیز از مقاومت مناسبی برخوردار است. از روش ارائه‌شده برای کنترل یک استند سه درجه آزادی چهارپره که به نوعی یک آونگ معكوس نیز هست، استفاده شده‌است. 
عملكرد این روش با اجرای شبیه‌سازی‌های مختلف مورد ارزیابی قرار خواهد گرفت. همچنین، عملكرد آن 
در حضور نویز و اغتشاش و عدم قطعیت مدل از طریق شبیهسازی ارزیابی خواهد‌شد.

\پرش‌بلند
\بدون‌تورفتگی \مهم{کلیدواژه‌ها}: 
معادله نش، استند سه درجه آزادی، شبیه‌سازی، تابع هزینه
\صفحه‌جدید
