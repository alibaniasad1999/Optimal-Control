\section{بازی همراه با بازخورد}
تفاوت بازی همراه با بازخورد \LTRfootnote{The Feeback Game} با بازی حلقه‌باز در این است که بازیکنان در هر لحظه از بازی بازخورد می‌گیرند و متانسب با بازخورد رفتار می‌کنند. این بازخورد ممکن است باعث شود یک بازیکن انگیزه پیدا کند که از بازی انحراف پیداکند در حالی که این اتفاق در بازی حلقه‌باز رخ نمی‌دهد. این اتفاق منجر به یک راه حل تعادلی دیگر می‌شود. از طرف دیگر راه حل تعادلی نباید در طول بازی خودش را با بازکنان سازگار کند.


با توجه به اینکه سیستم خطی است، می‌توان استدلال کرد که حرکات تعادل به صورت تابعی خطی از وضعیت سیستم است. این بدین مفهوم است که تعادل نش باید در فضای ذکر شده باشد. فضای استراتژی به فرم 

\begin{equation}\label{NashSpace}
	\Gamma^{lfb}_i :‌= \left\{u_i(0, T)\vert u_i(t) = F_i(t)x(t) ,~ i = 1, 2\right\}
\end{equation}
تعریف می‌شود. در رابطه
 \ref{NashSpace}
$F_i(.)$
قسمتی از یک تابع است. حرکات تعادل نش
$(u_1^*, u_2^*)$
در فضای استراتژی 
$\Gamma^{lfb}_1\times\Gamma^{lfb}_2$
است.
\شروع{قضیه} 
مجموعه‌ی حرکات کنترلی 
$u_i^*(t)=F_i^*(t)x(t)$
تشکیل شده‌است از بازخورد خطی تعادل نش اگر
\begin{equation*}
	J_1(u_1^*, u_2^*)\leq J_1(u_1, u_2^*)~ and~
	J_1(u_1^*, u_2^*)\leq J_1(u_1^*, u_2)
\end{equation*}
برای هر 
$u_i\in \Gamma^{lfb}_i$
برقرار باشد.
\پایان{قضیه}
\شروع{قضیه}
بازی دیفرانسیل خطی درجه دوم دو نفره برای هر شرایط اولیه، تعادل نش خطی بازخورد دارد اگر و فقط اگر مجموعه معادلات کوپل ریکاتی

\begin{align}
	\begin{split}
		\dot{K}_1(t) &= -(A-S_2K_2(t))^TK_1(t)-K_1(t)(A-S_2K_2(t))+
		K_1(t)S_1K_1(t)-Q_1\\
		\quad K_1(T) &= H_1
	\end{split}\\
	\begin{split}
		\dot{K}_2(t) &= -(A-S_1K_1(t))^TK_2(t)-K_2(t)(A-S_1K_1(t))+
		K_2(t)S_2K_2(t)-Q_2\\
		\quad K_2(T) &= H_2 
	\end{split}
\end{align}
در 
$[0, T]$
جواب متقارن داشته‌باشند(برای سادگی 
$S_{12}=S{21} =0 $
فرض شده‌است).
در این حالت دارای تعادل منحصر به فرد است. حرکت‌های تعادله به فرم رابطه
\ref{nash_action}
است.
\begin{equation}\label{nash_action}
	u_i^*(t) = -R_{ii}B_i^TK_i(T)x(T),~i = 1, 2
\end{equation}
\پایان{قضیه}
