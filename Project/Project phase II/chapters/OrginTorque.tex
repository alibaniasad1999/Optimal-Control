\section{گشتاور ناشی از نیروی تكیه‌گاه}
همانطور که در شكل \ref{QuadAssum} مشاهده می‌شود، نیروی
$f_d$
که در نقطه‌ی 
$D$
از طرق اتصال کلی به چهارپره وارد می‌شود، باعث ایجاد گشتاوری حول مرکز ثقل چهارپره می‌شود. به منظور مدل‌سازی گشتاور ناشی از این نیرو حول نقطه
$C$
، لازم است ابتدا نیروی
$f_d$
استخراج شود. از انجایی که نقطه‌ی
$D$
منطبق بر مرکز ثقل چهارپره نیست؛ لذا معادله حرکت انتقالی برای نقطه اتصال
$D$
با استفاده از معادله انتقال یافته نیوتن (معادله گروبین) به صورت معادله زیر حاصل می‌شود
\cite{zipfel2000modeling}
:
\begin{equation}\label{oprigintorque}
	m_{tot} \left[D^Iv_D^I\right]^B = 
	\left[\Sigma f\right]^B-m_{tot}\left\{
	\left[\Omega^{BI}\right]^B
	\left[\Omega^{BI}\right]^B
	\left[s_{cd}\right]^B+
	\left[D^I\Omega^{BI}\right]^B
	\left[s_{cd}\right]^B
	\right\}
\end{equation}
در رابطه
\ref{oprigintorque}، 
$m_{tot}$
مجموع جرم چهارپره و 
$\left[D^Iv_D^I\right]^B$
مشتق دورانی سرعت نقطه
$D$
نسبت به قاب اینرسی در دستگاه مختصات بدنی است. همچنین
$\left[\Sigma f\right]^B$
بیان‌کننده برآیند نیروهای وارده بر نقطه‌ی
$D$
و
$\left[D^I\Omega^{BI}\right]^B$
ماتریس پادمتقارن بردار سرعت زاوی‌های چهارپره نسبت به قاب اینرسی در دستگاه مختصات بدنی است. همچنین
$\left[D^I\Omega^{BI}\right]^B$
نشان‌دهنده مشتق دورانی سرعت زاوی‌های چهارپره نسبت به قاب اینرسی و 
$\left[s_{cd}\right]^B$
بردار واصل از نقطه‌ی
$D$
به نقطه
$C$
با انتقال قاب بدنی به قاب اینرسی، معادله 
\ref{oprigintorque}
به صورت زیر حاصل می‌شود:
\begin{equation}
\begin{split}\label{newoprigintorque}
	&m_{tot} \left[D^Bv_D^I\right]^B +
	m_{tot}\left[\Omega^{BI}\right]^B
	\left[v_D^{I}\right]^B = \\
	&\left[\Sigma f\right]^B-m_{tot}\left\{2
	\left[\Omega^{BI}\right]^B
	\left[\Omega^{BI}\right]^B
	\left[s_{cd}\right]^B+
	\left[D^I\Omega^{BI}\right]^B
	\left[s_{cd}\right]^B
	\right\}
\end{split}
\end{equation}
همچنین به دلیل اینكه سرعت محل اتصال چهارپره(نقطه
$D$)
صفر است؛ دو عبارت سمت چپ معادله
\ref{newoprigintorque} 
هر دو صفر هستند. در نتیجه معادله به صورت زیر ساده می‌شود.
\begin{equation}\label{newnewoprigintorque}
	\left[\Sigma f\right]^B - 
	m_{tot}\left\{2
	\left[\Omega^{BI}\right]^B
	\left[\Omega^{BI}\right]^B
	\left[s_{cd}\right]^B+
	\left[\dfrac{d\Omega^{BI}}{dt}\right]^B
	\left[s_{cd}\right]^B
	\right\}
\end{equation}
عبارت 
$\left[\Sigma f\right]^B$
بیانگر مجموع نیروهای وارد بر چهارپره است که به صورت معادله زیر بیان می‌شود:
\begin{equation}\label{sigmaF}
	\left[\Sigma f\right]^B = \left[f_D\right]^B+\left[f_T\right]^B+
	\left[f_G\right]^B
\end{equation}
در رابطه 
\ref{sigmaF}، بردار 
$\left[f_D\right]^B$
مقدار نیروی اعمال‌ شده توسط اتصال کلی در نقطه‌ی
$D$
است. همچنین  بردار 
$\left[f_T\right]^B$
بیانگر مدموع نیروی تراست پرهها در دستگاه مختصات بدنی است که از رابطه زیر حاصل می‌شود:
\begin{equation}\label{trustmatrix}
	\left[f_G\right]^B = \begin{bmatrix}
		0\\0\\
		T_1+T_2+T_3+T_4
	\end{bmatrix}
\end{equation}
مقدار نیروی اعمال‌ شده توسط اتصال کلی در نقطه‌ی
$D$
است. همچنین  بردار 
$\left[f_G\right]^B$
بیانگر نیروی وزن چهارپره در دستگاه مختصات بدنی است که از رابطه زیر حاصل می‌شود:
\begin{equation}\label{fg}
	\left[f_G\right]^B = \left[C\right]^{BL}
	\left[f_G\right]^L
\end{equation}
در رابطه
\ref{fg}،
ماتریس انتقال از دستگاه مختصات تراز محلی
$(L)$
 به دستگاه مختصات بدنی است. با جایگذاری روابط
\ref{newnewoprigintorque}،
\ref{sigmaF}،
\ref{trustmatrix}و
\ref{fg}
عبارت زیر برای نیروی تكیه‌گاهی حاصل می‌شود.
\begin{equation}\label{originnew}
	\left[f_D\right]^B = 
	-\left[f_G\right]^B-
	\left[f_T\right]^B+
	m_{tot}\left\{2
	\left[\Omega^{BI}\right]^B
	\left[\Omega^{BI}\right]^B
	\left[s_{cd}\right]^B+
	\left[\dfrac{d\Omega^{BI}}{dt}\right]^B
	\left[s_{cd}\right]^B
	\right\}
\end{equation}
سپس از حاصل‌ضرب نیروی تكیه‌گاه مدل‌ شده در معادله
\ref{originnew}
 در بردار محل اثر آن، گشتاور ایجاد شده
توسط نیروی اتصال کلی به صورت معادله زیر حاصل می‌شود:
\begin{equation}\label{universaltor}
	\left[m_d\right]^B = \left[s_{DC}\right]^B\left(
	-\left[f_G\right]^B
	-\left[f_T\right]^B
	m_{tot}\left\{2
	\left[\Omega^{BI}\right]^B
	\left[\Omega^{BI}\right]^B
	\left[s_{cd}\right]^B
	\right\}
	\right)
\end{equation}
در رابطه
\ref{universaltor}
بردار 
$\left[s_{DC}\right]^B$
بیانگر فاصله‌ی نقطه‌ی 
$D$
از مرکز ثقل چهارپره 
$(h_{cg})$
است که به صورت زیر بیان می‌شود:
\begin{equation}
	\left[s_{DC}\right]^B = \begin{bmatrix}
		0\\0\\h_{cg}
	\end{bmatrix}
\end{equation}
 درنتیجه با جمع گشتاورهای ناشی از نیروهای آیرودینامیک پره‌ها از معادله 
 \ref{finaltorque}
 و گشتاور ناشی 
از نیروی تكیه‌گاه از معادله 
\ref{universaltor}، گشتاور خارجی کلی اعمالی به چهارپره به صورت معادله زیر حاصل 
می‌شود:
\begin{equation}\label{finalm}
	\left[m_B\right]^B = 
	\left[m_A\right]^B+
	\left[m_D\right]^B
\end{equation}

