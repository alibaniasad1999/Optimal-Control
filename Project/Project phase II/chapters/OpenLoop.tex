\section{بازی حلقه‌باز}
در این حالت فرض شده است که تمامی بازیکنان در زمان 
$t \in [0, T]$
فقط اطلاعات شرایط اولیه و مدل سیستم را دارند. این فرض به این صورت تفسیر می‌شود که دو بازیکن همزمان حرکت خود را در انتخاب می‌کنند. در این حالت امکان بستن قرارداد بین دو بازیکن وجود ندارد. تعادل نش در ادامه تعریف شده‌است.
\شروع{قضیه} 
به مجموعه‌ای از حرکات قابل قبول 
$(u_1^*, u_2^*)$
یک \مهم{تعادل نش} برای بازی می‌گویند اگر تمامی حرکات حرکات قابل قبول 
$(u_1, u_2)$
از نامساوی\ref{nash_lqe} پیروی کنند.
\begin{equation}\label{nash_lqe}
	J_1(u_1^*, u_2^*)\leq J_1(u_1, u_2^*)~ and~
	J_1(u_1^*, u_2^*)\leq J_1(u_1^*, u_2)
\end{equation}
\پایان{قضیه}
در اینجا قابل قبول بودن به معنی است که
$u_i(.)$
به یک مجموعه محدود حرکات تعلق دارد، این مجموعه حرکات که بستگی به بازیکنان اطلاعات از بازی دارد، مجموعه‌ای از استراتژی‌هایی که بازیکنان دوست دارند برای کنترل سیستم انجام دهند و سیستم 
\ref{system_dynamic}
باید یک جواب منحصر به فرد داشته باشد. 


تعادل نش به گونه‌ای تعریف می‌شود که هیچ یک از بازیکنان انگیزه‌ی یک طرفه برای انحراف از بازی ندارند. قابل ذکر است که نمی‌توان انتظار داشت که یک تعادل نش منحصر به فرد وجود داشته باشد. به هر حال به راحتی می‌توان تایید کرد که حرکات
$(u_1^*, u_2^*)$
یک تعادل نش برای بازی با تابع هزینه
$J_i,~ i = 1, 2$
است. اگر تعادل نش برای تابع هزینه قسمت قبل برقرار باشد برای تایع هزینه
$\alpha_iJ_i,~ i = 1, 2, ~\alpha_i>0$
نیز برقرار است.


برای سادگی از نمادسازی 
$S_i := B_iR^{-1}_{ii}B_i^T$
استفاده شده‌است. در اینجا فرض شده که زمان $T$ محدود است.
\شروع{قضیه} \label{openlooptheorm}
ماتریس $M$ را در نظر بگیرید:
\begin{equation}
	M :=
	\begin{bmatrix}
		A & -S_1 & -S_2\\
		-Q_1 & -A^T& 0\\
		-Q_2 & 0 & -A^T
	\end{bmatrix}
\end{equation}
فرض شده‌است که دو معادله دفرانسیلی ریکاتی
(\ref{riccati_teorm})
،$K_i(i)$
در بازه
$[0, T]$
جواب متقارن دارند.
\begin{equation}\label{riccati_teorm}
	\dot{K}_i(t) = -A^TK_i(t)-K_i(t)A+K_i(t)S_iK_i(t)-Q_i,\quad K_i(T) = H,\quad i = i, 2
\end{equation}
\newpage
پس بازی دیفرانسیل خطی درجه دوم\LTRfootnote{the two player linear quadratic differential game} دو نفره دارای تعادل نش حلقه‌باز در هر شرایط اولیه $X_0
$
دارد اگر ماتریس
\begin{equation}
	H(T) := \begin{bmatrix}
		I&0&0
	\end{bmatrix}
e^{-MT}
\begin{bmatrix}
	I\\Q_{1T}\\Q_{2T}
\end{bmatrix}
\end{equation}
قابلیت معکوس شدن را  داشته‌ باشد.
\پایان{قضیه}
در آخر با استفاده از قضیه
 \ref{openlooptheorm}
با حل دو معادله کوپل ریکاتی دیفرانسیلی می‌توان به جواب رسید.
\begin{align}
	\dot{P}_1 &= -A^TP_1 - P_1A - Q_1 +P_1S_1P_1 + P_1S_2P_2;\quad P_1(T) = H_1\\
	\dot{P}_2 &= -A^TP_2 - P_2A - Q_2 +P_2S_2P_2 + P_2S_1P_1;\quad P_2(T) = H_2
\end{align}

