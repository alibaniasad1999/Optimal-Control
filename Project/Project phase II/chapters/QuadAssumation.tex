\section{فرضیات مدل‌سازی}
شماتیک استند چهارپره در شكل \ref{QuadAssum}نشان داده شده‌است. به ‌منظور استخراج معادلات حاکم بر سیستم، 
فرض می‌شود که چهارپره صلب و متقارن است. همچنین ماتریس گشتاور اینرسی چهارپره به صورت قطری درنظر گرفته می‌شود. مرکز ثقل سازه چهارپره روی نقطه $B$ و مرکز ثقل هر یک از پره‌ها به همراه قسمت دوار موتور روی نقاط 
$B_1$
تا
$B_2$
است. مبدأ دستگاه مختصات بدنی روی محل تقاطع بازوهای چهارپره یعنی نقطه 
$B$
در نظر گرفته‌شده‌است. از آنجایی ‌که مرکز ثقل پره‌ها بالاتر از مرکز ثقل سازه چهارپره است، مرکز ثقل کلی چهارپره جایی بین مرکز ثقل موتورها و سازه، یعنی نقطه‌ی 
$C$
می‌گیرد. همچنین قابل ذکر است که نقطه‌ی
$D$
محل اتصال کلی استند چهارپره است. جهت مثبت محور 
$X^B$
و
$Y^B$
دستگاه مختصات بدنی به ترتیب در راستای بازوی مربوط به موتور 1 و 4 فرض می‌شود. همچنین جهت مثبت محور
$Z^B$
با توجه به قانون دست راست حاصل می‌شود.
\begin{figure}[H]\label{QuadAssum}
	\includegraphics[width=12cm]{figs/Quad/StandAssumations.jpg}
	\centering
	\caption{شماتیک استند چهارپره\cite{Abeshtan}}
\end{figure}
 