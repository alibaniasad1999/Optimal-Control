\section{خطی‌سازی}
در این قسمت با توجه به فضای حالت بدست آمده، چهارپره حول نقطه کار خطی‌سازی می‌شود.
\begin{equation*}
	a = \begin{bmatrix}
		x_4 + x_5\sin(x_1)\tan(x_2) + x_6\cos(x_1)\tan(x_2)\\
		x_5\cos(x_1)- x_6\sin(x_1)\\
		(x_5\sin(x_1) + x_6\cos(x_1))\sec(x_2)\\
		A_1\cos(x_2)\sin(x_1) + 
		A_2x_5x_6 + A_3\left(\omega_2^2-\omega_4^2\right)+
		A_4x_5\left(\omega_1-\omega_2+\omega_3-\omega_4\right)\\
		B_1\sin(x_2) + 
		B_2x_4x_6 + B_3\left(\omega_1^2-\omega_3^2\right)+
		B_4x_4\left(\omega_1-\omega_2+\omega_3-\omega_4\right)\\
		C_1x_4x_5 + 
		C_2\left(\omega_1-\omega_2+\omega_3-\omega_4\right)
	\end{bmatrix}
\end{equation*} 
\begin{equation}
	\vec{x} = \begin{bmatrix}
		\phi& \theta & \psi & p& q& r
	\end{bmatrix}^T
\end{equation}
\begin{equation}
	\vec{u} = \begin{bmatrix}
		\omega_1&\omega_2&\omega_3&\omega_4
	\end{bmatrix}^T
\end{equation}
برای خطی سازی از بسط تیلور استفاده شده‌است.
\begin{equation}
	\delta \dot x = \dfrac{\partial a}{\partial x}\delta x + \dfrac{\partial a}{\partial u}\delta u 
\end{equation}
\begin{equation}
	\dot{\vec{x}} =
	\begin{bmatrix}
		\delta \dot x_1&
		\delta \dot x_2&
		\delta \dot x_3&
		\delta \dot x_4&
		\delta \dot x_5&
		\delta \dot x_6&
	\end{bmatrix}^T
\end{equation}

\begin{equation}
	\dfrac{\partial \vec a}{\partial \vec x} = A = 
	\begin{bmatrix}
\dfrac{\partial  a_1}{\partial  x_1}&
\dfrac{\partial  a_1}{\partial  x_2}&
\dfrac{\partial  a_1}{\partial  x_3}&
\dfrac{\partial  a_1}{\partial  x_4}&
\dfrac{\partial  a_1}{\partial  x_5}&
\dfrac{\partial  a_1}{\partial  x_6}&
\\[1em]
\dfrac{\partial  a_2}{\partial  x_1}&
\dfrac{\partial  a_2}{\partial  x_2}&
\dfrac{\partial  a_2}{\partial  x_3}&
\dfrac{\partial  a_2}{\partial  x_4}&
\dfrac{\partial  a_2}{\partial  x_5}&
\dfrac{\partial  a_2}{\partial  x_6}&
\\[1em]
\dfrac{\partial  a_3}{\partial  x_1}&
\dfrac{\partial  a_3}{\partial  x_2}&
\dfrac{\partial  a_3}{\partial  x_3}&
\dfrac{\partial  a_3}{\partial  x_4}&
\dfrac{\partial  a_3}{\partial  x_5}&
\dfrac{\partial  a_3}{\partial  x_6}&
\\[1em]
\dfrac{\partial  a_4}{\partial  x_1}&
\dfrac{\partial  a_4}{\partial  x_2}&
\dfrac{\partial  a_4}{\partial  x_3}&
\dfrac{\partial  a_4}{\partial  x_4}&
\dfrac{\partial  a_4}{\partial  x_5}&
\dfrac{\partial  a_4}{\partial  x_6}&
\\[1em]
\dfrac{\partial  a_5}{\partial  x_1}&
\dfrac{\partial  a_5}{\partial  x_2}&
\dfrac{\partial  a_5}{\partial  x_3}&
\dfrac{\partial  a_5}{\partial  x_4}&
\dfrac{\partial  a_5}{\partial  x_5}&
\dfrac{\partial  a_5}{\partial  x_6}&
\\[1em]
\dfrac{\partial  a_6}{\partial  x_1}&
\dfrac{\partial  a_6}{\partial  x_2}&
\dfrac{\partial  a_6}{\partial  x_3}&
\dfrac{\partial  a_6}{\partial  x_4}&
\dfrac{\partial  a_6}{\partial  x_5}&
\dfrac{\partial  a_6}{\partial  x_6}&
	\end{bmatrix}
\end{equation}
\begin{equation}
	\dfrac{\partial \vec a}{\partial \vec u} = B = 
	\begin{bmatrix}
		\dfrac{\partial  a_1}{\partial  u_1}&
		\dfrac{\partial  a_1}{\partial  u_2}&
		\dfrac{\partial  a_1}{\partial  u_3}&
		\dfrac{\partial  a_1}{\partial  u_4}&
		\\[1em]
		\dfrac{\partial  a_2}{\partial  u_1}&
		\dfrac{\partial  a_2}{\partial  u_2}&
		\dfrac{\partial  a_2}{\partial  u_3}&
		\dfrac{\partial  a_2}{\partial  u_4}&
		\\[1em]
		\dfrac{\partial  a_3}{\partial  u_1}&
		\dfrac{\partial  a_3}{\partial  u_2}&
		\dfrac{\partial  a_3}{\partial  u_3}&
		\dfrac{\partial  a_3}{\partial  u_4}&
		\\[1em]
		\dfrac{\partial  a_4}{\partial  u_1}&
		\dfrac{\partial  a_4}{\partial  u_2}&
		\dfrac{\partial  a_4}{\partial  u_3}&
		\dfrac{\partial  a_4}{\partial  u_4}&
		\\[1em]
		\dfrac{\partial  a_5}{\partial  u_1}&
		\dfrac{\partial  a_5}{\partial  u_2}&
		\dfrac{\partial  a_5}{\partial  u_3}&
		\dfrac{\partial  a_5}{\partial  u_4}&
		\\[1em]
		\dfrac{\partial  a_6}{\partial  u_1}&
		\dfrac{\partial  a_6}{\partial  u_2}&
		\dfrac{\partial  a_6}{\partial  u_3}&
		\dfrac{\partial  a_6}{\partial  u_4}&
		\\[1em]
	\end{bmatrix}
\end{equation}
به علت حجم بالای معادلات رابطه خطی‌سازده شده چهارپرده در گزایش آورده نشده‌است اما در شبیه سازی به طور کامل لحاظ شده‌است.
