%%%%%%%%%%%%%%%%%%%%%%%%%%%%%%%%%%%%%%%%%
% fphw Assignment
% LaTeX Template
% Version 1.0 (27/04/2019)
%
% This template originates from:
% https://www.LaTeXTemplates.com
%
% Authors:
% Class by Felipe Portales-Oliva (f.portales.oliva@gmail.com) with template 
% content and modifications by Vel (vel@LaTeXTemplates.com)
%
% Template (this file) License:
% CC BY-NC-SA 3.0 (http://creativecommons.org/licenses/by-nc-sa/3.0/)
%
%%%%%%%%%%%%%%%%%%%%%%%%%%%%%%%%%%%%%%%%%

%----------------------------------------------------------------------------------------
%	PACKAGES AND OTHER DOCUMENT CONFIGURATIONS
%----------------------------------------------------------------------------------------

\documentclass[12]{fphw}
\usepackage{float}
% Template-specific packages

\usepackage{graphicx} % Required for including images


\usepackage{fullpage,enumitem,amsmath,amssymb,graphicx}
\usepackage{pdfpages}
\usepackage{amsmath} 

%----------------------------------------------------------------------------------------
%	ASSIGNMENT INFORMATION
%----------------------------------------------------------------------------------------

\title{Homework \#1} % Assignment title

\author{Ali BaniAsad} % Student name

\date{March 28th, 2025} % Due date

\institute{Sharif University of Technology \\ Institute of Aerospace} % Institute or school name

\class{Optimal Control I} % Course or class name

\professor{Dr. Assadian} % Professor or teacher in charge of the assignment

%----------------------------------------------------------------------------------------

\begin{document}

\maketitle % Output the assignment title, created automatically using the information in the custom commands above
\section*{Problem 1}

\begin{enumerate}[label=(\alph*)]
	\item 
	$z = f(x, y) = y\sin(x+y)-x\sin(x-y)$ \\
Gradient of f(x, y):
$$\vec{\nabla} f = \begin{bmatrix}
	\frac{\partial f}{\partial x} \\[6pt]
	\frac{\partial f}{\partial y}
\end{bmatrix} $$
$$\vec{\nabla} f = \begin{bmatrix}
	y \cos(x + y) - \sin(x - y) - x  \cos(x - y) \\
	y  \cos(x + y) + \sin(x + y) + x  \cos(x - y)
\end{bmatrix} $$

non linear equations with two unknowns. We use MATLAB to solve this equations. MATLAB file is attached.
\begin{table}[h]
				\caption {Answers} \label{ans} 
	\begin{center}
		\begin{tabular}{| l | l |}
			\hline
			x & y\\ \hline
			-3.41877 & -1.82764 \\ \hline
			-2.88904 & 1.84693 \\ \hline
			-2.02875 & 0.00000 \\ \hline
			-1.84693 & -2.88904 \\ \hline
			-1.82764 & 3.41877 \\ \hline
			-1.75560 & 0.36547 \\  \hline
			-0.36547 & -1.7556 \\ \hline
			0.00000 & -2.02875 \\ \hline
			0.00000 & 0.00000 \\ \hline
			0.00000 & 2.02875 \\ \hline
			0.36547 & 1.7556 \\ \hline
			1.75560 & -0.36547 \\ \hline
			1.82764 & -3.41877 \\ \hline
			1.84693 & 2.88904 \\ \hline
			2.02875 & 0.00000 \\ \hline
			2.88904 & -1.84693 \\ \hline
			3.41877 & 1.82764 \\ \hline
		\end{tabular}
	\end{center}
\end{table}
Answers are provided in table \ref{ans}



	\newpage
	\item 
	$z = f(x, y) = x^3 - 3xy^2$ \\
Gradient of f(x, y):
$$\vec{\nabla} f = \begin{bmatrix}
	\dfrac{\partial f}{\partial x} \\[6pt]
	\dfrac{\partial f}{\partial y}
\end{bmatrix} $$
$$\vec{\nabla} f = \begin{bmatrix}
	3x^2 - 3y^2 \\
	-6xy
\end{bmatrix} $$
Two linear equations with two unknowns.
$$	3x^2 - 3y^2 =  0 $$
$$-6xy = 0$$
Answers is $x = 0$ and $y = 0$.


Hessian matrix:
$$H = \dfrac{\partial^2 f}{\partial \vec{X}^2} = \begin{bmatrix}
	\dfrac{\partial^2 f}{\partial x^2} & \dfrac{\partial^2 f}{\partial xy} \\[6pt]
	\dfrac{\partial^2 f}{\partial yx}  & \dfrac{\partial f}{\partial y^2}
\end{bmatrix} $$
$$H = \begin{bmatrix}
	6x & -6y \\
   -6y & -6x
\end{bmatrix}$$



In $x = 0$ and $y = 0$ Hessian matrix in :
$$H = \begin{bmatrix}
	0 & 0 \\
	0 & 0
\end{bmatrix}$$
so this point is saddle point.
\begin{figure}[H]
	\caption{3D figure of function}
	\centering
	\includegraphics[width=12cm]{Q1/figures/3DplotQ1b.png}
\end{figure}

\begin{figure}[H]
	\caption{3D figure of function with Points}
	\centering
	\includegraphics[width=12cm]{Q1/figures/3DplotWithPointsTashihQ1b.png}
\end{figure}

\begin{figure}[H]
	\caption{Contour figure of function}
	\centering
	\includegraphics[width=12cm]{Q1/figures/ContourQ1b.png}
\end{figure}
\begin{figure}[H]
	\caption{Contour figure of function with Points}
	\centering
	\includegraphics[width=12cm]{Q1/figures/ContourWithPointsQ1b.png}
\end{figure}

	\newpage
	\item 
	$z = f(x_1, x_2, x_3) = x_1^2 + x_1x_2 - 4x_2^2-x_3^2 + 3x_2x_3$ \\
Gradient of $f(x_1, x_2, x_3)$:
$$\vec{\nabla} f = \dfrac{\partial f}{\partial \vec{X}}= \begin{bmatrix}
	\dfrac{\partial f}{\partial x_1} \\[8pt]
	\dfrac{\partial f}{\partial x_2} \\[8pt]
	\dfrac{\partial f}{\partial x_3} \\
\end{bmatrix} $$
$$\vec{\nabla} f =   \begin{bmatrix}
	2x_1 + x_2 \\
	x_1 -8x_2 + 3x_3 \\
	3x_2 - 2x_3
\end{bmatrix}  = \vec{0}$$
Three linear equations with Three unknowns.
$$	2x_1 + x_2 = 0 $$
$$x_1 -8x_2 + 3x_3 = 0$$
$$3x_2 - 2x_3 = 0 $$
Answers is $x_1 = x_2 = x_3 = 0$
Hessian matrix:
$$H = \dfrac{\partial^2 f}{\partial \vec{X}^2} = \begin{bmatrix}
	\dfrac{\partial^2 f}{\partial x_!^2} & \dfrac{\partial^2 f}{\partial x_1x_2} & \dfrac{\partial^2 f}{\partial x_1x_3} \\[8pt]
	\dfrac{\partial^2 f}{\partial x_2x_1}  & \dfrac{\partial f}{\partial x_2^2} & \dfrac{\partial^2 f}{\partial x_2x_3}  \\[8pt]
		\dfrac{\partial^2 f}{\partial x_3x_1}  & \dfrac{\partial f}{\partial x_3x_2} & \dfrac{\partial^2 f}{\partial x_3^2} 
\end{bmatrix} $$
$$H = \begin{bmatrix}
	2 & 1 & 0 \\
	1 & -8 & 3 \\
	0 & 3 & -2
\end{bmatrix}$$
All Hessian eigenvalues are:
$$eig(H) = \begin{bmatrix}
	-9.3182  \\
	-0.8077 \\
	2.1259
\end{bmatrix}$$
So $(0, 0, 0)$ is a saddle point.
\end{enumerate}
\newpage
\section*{Problem 2}
$$\min f(x, y, z) = x^2 + y^2 + z^ 2$$
subject to : $$z = \sin(x) + \cos(y)$$
\begin{enumerate}[label=(\alph*)]
	\item 
	Direct Substitution:


$z = \sin(x) + \cos(y) \xrightarrow{f(x, y, z) = x^2 + y^2 + z^ 2}
f(\vec{X}) = f(x, y) = x^2 + y^2 + (\sin(x)+\cos(y))^2
$


$f(x, y) = x^2 + y^2 + \sin(x)^2 + 2\sin(x)\cos(y) + \cos(y)^2$
Gradient of f(x, y):
$$\vec{\nabla} f = \begin{bmatrix}
	\dfrac{\partial f}{\partial x} \\[6pt]
	\dfrac{\partial f}{\partial y}
\end{bmatrix} $$
$$\vec{\nabla} f = \begin{bmatrix}
	2x + 2\cos(x)\cos(y) + 2\cos(x)\sin(x) \\
	2y - 2\cos(y)\sin(y) - 2\sin(x)\sin(y)
\end{bmatrix} $$
$$	2x + 2\cos(x)\cos(y) + 2\cos(x)\sin(x)=  0 $$
$$2y - 2\cos(y)\sin(y) - 2\sin(x)\sin(y)= 0$$
Above equation solved in MATLAB and code (Q2\_a.m) has attached to homework.


$x = -0.47872, \quad y = 0.0 \to z = 0.5393 $
\begin{table}[H]
	\caption {Answers} \label{ans} 
	\begin{center}
		\begin{tabular}{| l | l | l |}
			\hline
			x & y & z \TBstrut \\
			\hline
			-0.47872 & 0.000 & 0.5393 \Tstrut\\
			\hline
		\end{tabular}
	\end{center}
\end{table}



\end{enumerate}
\end{document}
